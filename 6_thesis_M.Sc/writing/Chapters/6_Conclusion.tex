\chapter{Conclusion}
\label{sec:conclusion}

This study aimed at evaluating the integration of the \gls{osm} road network and two of its attributes, road types and speed limits, into the feature space of a semantic segmentation model for \gls{lulcm}. Hypothetically, by adding information about the spatial context of \gls{lulc} classes using a road network, the model convergence should be supported and the overall resource consumption reduced alongside an increase of the performance, as the model can make more informed decisions about the \gls{lulc} classes. The feature engineered attributes were encoded via \gls{ohe} into the feature space of the model. The results revealed that this integration provides valuable spatial contextual information about adjacent \gls{lulc} classes, primarily increasing the model's performance while needing less epochs for convergence. Therefore, the model can make more informed decisions about the prediction of \gls{lulc} areas, partially confirming the hypothesis. In four out of six \gls{lulc} classes and five out of six metric comparison cases, the extensions outperform the baseline. This improvement, however, comes at the cost of increased resource consumption and longer runtimes, highlighting a trade-off between enhanced model performance and resource efficiency. Still, the results showed that the model could potentially achieve similar performance with a higher efficiency and less resource consumption, if it is optimized accordingly. The study also highlighted that increasing the complexity of the model by adding more information does not guarantee better performance unless the information is optimized. This finding emphasizes the need for careful consideration of the information included in the model to balance performance gains and resource consumption. Future research should focus on optimizing the model to reduce resource consumption while maintaining or enhancing performance. Applying advanced optimization techniques and fine-tuning hyperparameters are expected to improve the model's efficiency and sustainability and keep the level of increased performance, if not even increased. Evaluating these optimizations with the full feature space of the vanilla model could further reveal the true potential of road network data as an enhancement for \gls{lulcm} models.

The proof-of-concept that this study aimed for was achieved, but the results are not generalizable due to the small dataset and the specific setup. The study should be repeated with a larger dataset, the omission of the \gls{lcbb}, a further reduction of road network attributes to only tags with substantial relations to specific \gls{lulc} classes, adapted \gls{lulc} filters or other \gls{lulc} classes, fine-tuned hyperparameters for the extended feature space, and according optimization techniques. This would allow for a more comprehensive evaluation of the integration of road network data into \gls{lulcm} models and provide a more reliable basis for future research in this field. However, it is expected that the tendencies regarding an increase in performance and a possible reduction of resource consumption will remain similar, as the results of this study are in line with previous research on the integration of contextual data into models for \gls{lulcm}. Also, the methodology is regarded feasible as it is based on established practices and tools as well as prior studies, and the results are reproducible. As other studies have shown, \gls{osm} data is suitable for training and enhancing \gls{dl} models for various tasks, including the integration of spatial contextual information for effective and efficient \gls{lulcm}. But, the data has to be preprocessed and feature engineered carefully to avoid anomalies and irrelevant information that could decrease the model's performance.

Alongside multiple other studies, this study has proven that enriching \gls{dl} models with contextual data has the potential to increase the performance while making them more efficient and sustainable. In regard to climate change and the \glspl{sdg}, this is a crucial step towards reducing their negative environmental impacts and utilizing their full potential sustainably. For future research, this should be the main aim, to develop sustainable \gls{dl} models that can be applied to various tasks and domains, as they have major potential for human development in any regard. This study has shown that the integration of road network data into \gls{lulcm} models is a promising approach to achieve this goal in one small field of the large landscape of applications of \gls{dl}.