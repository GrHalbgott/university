\chapter{Introduction}
\label{sec:intro}

% AI super current with ChatGPT and LLMs
It's everywhere: in the news, in research and science, on the web, in our hands, used in our daily lives as a companion, its usage strongly debated in universities and schools, and its possibilities seemingly endless \autocite{Roumeliotis.Tselikas2023}. \gls{ai} is currently one of the most discussed technological topics thanks to the development of large language models and the emergence of {\nobreak OpenAI's} famous ChatGPT (Generative Pre-Trained Transformer, \autocite{OpenAI2022}) towards the end of 2022, functioning as a chatbot successfully mimicking human conversation \autocite{Abdullah.Madain.ea2022,Alzubaidi.Zhang.ea2021}. Despite its term being coined as far back as 1956 during the Dartmouth Conferences, it is the recent improvements in computational power, networks, and the advent of big data that have truly advanced its growth and application \autocite{Ongsulee2017}. \gls{ai} has already reinvented numerous sectors, from healthcare to finance, by automating tasks, improving decision-making processes, and providing new insights from data \autocite{Alzubaidi.Zhang.ea2021,Bini2018,Ongsulee2017,Roumeliotis.Tselikas2023}.

% ML and Climate Change, SDGs
\gls{ml}, a subfield of \gls{ai} that enables data-driven pattern recognition and sits at the intersection of \gls{ai} and data science, has emerged as a powerful tool for global sustainable development. One of the most critical issues in the 21st century, it aims at fulfilling current requirements of human development without compromising future generations \autocite{UnitedNations2023}. \gls{ml} provides innovative solutions for monitoring and managing natural resources, predicting environmental changes, and driving sustainable practices across various sectors \autocite{Getzner.Charpentier.ea2023,Kar.Choudhary.ea2022,Mehlin.Schacht.ea2023}. Goal 13 \enquote{Climate Action} of the \glspl{sdg}, set by the United Nations to be achieved by 2030, explicitly addresses the need for urgent action to combat climate change and its impacts. Furthermore, goal 11 \enquote{Sustainable Cities and Communities} aims to make cities inclusive, safe, resilient, and sustainable, by addressing comprehensive urban issues such as infrastructure, management, and urban planning, a goal including the advancements in \gls{ai} and, specifically, \gls{ml} and its possibilities for sustainable development \autocite{UnitedNations2023}.

% LULC-M and IPCC
An important geographical research field between \gls{ai} and sustainability is the utilization of \gls{ml} models for \gls{lulcm}, which gains a lot of attention from the need for sustainable development and urban planning \autocite{Zhao.Tu.ea2023}. The Intergovernmental Panel on Climate Change states that \enquote{global greenhouse gas emissions have continued to increase, with unequal historical and ongoing contributions arising from unsustainable energy use, land use and land use change, lifestyles and patterns of consumption and production across regions, between and within countries, and among individuals} \autocite[4]{IPCC2023}. \gls{lulc} is named explicitly as one of the key sectors where more sustainable interaction methods are needed to mitigate climate change, showing its importance in the context of sustainable development.

% What is LULC-M
\gls{lulcm} segments the landscape into patches of different land cover and land use types which gives valuable insights into their distribution and interaction at a given time \autocite{Li.Cai.ea2024}. Its largest potential lies in monitoring human-made and natural landscape changes which enables stakeholders to adapt the anthropogenic land uses in order to act more sustainable in and with the environment \autocite{Moharram.Sundaram2023}. Moreover, it plays a crucial role in diverse sectors such as disaster management, monitoring carbon emissions, and agricultural production, which underlines its broad spectrum of applications \autocite{Alhassan.Henry.ea2020,Rangel.Terven.ea2024,Zhao.Tu.ea2023}.

% Deep Learning for LULC-M
Initially done manually or semi-automated, \gls{lulcm} has been revolutionized by the advent of \gls{ml} and \gls{dl} models, especially image segmentation models, which have significantly improved the accuracy and efficiency of the process \autocite{Zhao.Tu.ea2023}. These models are trained on large datasets of satellite imagery to predict the \gls{lulc} classes of a given area, which can then be used for further analyses like monitoring changes over time and space \autocite{Li.Cai.ea2024}. Especially \gls{dl} models building on the Transformer, serving as the foundation for ChatGPT, have shown great potential in this field due to their ability to capture long-range dependencies and contextual information in images by utilizing attention mechanisms, partially enlarging receptive fields beyond their initial fixed reach \autocite{Alhichri.Alswayed.ea2021,Chen.Yang.ea2016,Xie.Wang.ea2021,Zheng.Lu.ea2021}.

% Environmental impact of ML
However, development and operation of \gls{ml} models require substantial computational resources and energy, which adds negatively to their environmental benefits. This is in particular true for \gls{dl} models as training increasingly complex models for improved accuracy results in greater energy use and carbon emissions \autocite{Desislavov.Martinez-Plumed.ea2023,Mehlin.Schacht.ea2023,Strubell.Ganesh.ea2019}. It is estimated that training a large Transformer model with 213 Million parameters (like a GPT) emits as much carbon dioxide as an average car drive for nearly 400 Kilometers \autocite{Strubell.Ganesh.ea2019,CSS2023}. This shows that the environmental impact of \gls{ml} models is not to be underestimated and that their sustainability has to be addressed alongside improving solely their performance. This ensures the technology's overall viability in the long term, also in respect to the named \glspl{sdg} \autocite{Getzner.Charpentier.ea2023,Thompson.Greenewald.ea2022}.

% Two ways to enhance ML models
So, as the power of \gls{ai} is harnessed for sustainable development, there are two main ways to enhance \gls{ml} and \gls{dl} models: either enhancing the performance by utilizing model improvements or reducing the resource consumption by utilizing sustainable methods for existing models \autocite{Getzner.Charpentier.ea2023,Moharram.Sundaram2023,Zhao.Tu.ea2023}. This is a constant goal in the field of \gls{ml} and \gls{dl} \autocite{Ongsulee2017,Roumeliotis.Tselikas2023}. In order to reduce the consumption of resources, a key strategy involves supporting model convergence with intelligent adaptations along with keeping the accuracies at a similar level, thereby shifting from a \enquote{faster-larger-better} to a \enquote{similar-but-more-efficient} approach \autocite{Lazzaro.Cina.ea2023,Mehlin.Schacht.ea2023,Tao.Meng.ea2022}. But, such a strategy can also enhance the accuracy of the models, as the models can make more informed decisions about the data they are trained on \autocite{Mehlin.Schacht.ea2023}.

% Contextual information
Based on the possibilities brought by the attention mechanisms, larger receptive fields, and the idea of multiple studies, it seems that in the field of \gls{lulcm}, providing spatial contextual information to \gls{dl} models is able to fulfill both roles. \textcite{Zhao.Ning.ea2023} describe the main concept of spatial context using Tobler's First Law of Geography \autocite{Tobler1970}: \enquote{The spatial distribution of geographical things or attributes is interrelated with each other, and it appears with clustering, random, and regular characteristics. The relationship can be described by the spatial context. Hence, spatial context extraction from the surrounding environment of geographic objects can enrich their characteristics and is very useful for predicting their types} \autocite[2]{Zhao.Ning.ea2023}. In their study, they used multiple spatial contexts like intersections, surrounding buildings, and points of interest to predict road types in satellite imagery. Additionally, they used tags from \gls{osm} which added information about traffic behavior and therefore the importance of roads, helping classifying the hierarchically structured road types. \textcite{Xu.Zhou.ea2022,Zong.He.ea2020} also utilized context inclusion, including points of interest and nighttime lights along with other data, to enhance \gls{lulcm}. They and others additionally used road networks from \gls{osm} to divide their \glspl{aoi} into research units and urban segments. \textcite{Forget.Linard.ea2018} went even a step further and assumed a correlation between road network and adjacent urban blocks as well as building types in order to enhance the prediction of built-up areas, an assumption similarly established by \textcite{Ahmadzai2020}. They all showed that including spatial contextual information greatly helps in prediction and can be used to enhance the performance of models, as proposed in its essence as early as in Tobler's First Law of Geography in 1970.

% Aim
This raises the question: if a road network provides additional attributes like road types, can it be utilized to enhance the mapping of all the surrounding \gls{lulc} areas, thus, \gls{lulcm}? This could be a valuable addition to the model, as roads are often built with adjacent \gls{lulc} classes in mind and, therefore, should give hints towards them \autocite{Ahmadzai2020,Levinson.Xie.ea2007,Zeng.Zhao.ea2020}. 

The main task of this study is the evaluation of a contextual inclusion approach which involves the integration of a road network into the feature space of a \gls{lulcm} model through feature engineering. The aim is to enrich the model with valuable spatial contextual information about the adjacent \gls{lulc} classes, given their specific geographical positioning within the landscape and relationship to the road network. Hypothetically, by adding information about the spatial context of \gls{lulc} classes using a road network, the model convergence should be supported and the overall resource consumption reduced alongside an increase of the performance, as the model can make more informed decisions about the \gls{lulc} classes.

% Framework
The \gls{dl} framework selected for assessing this hypothesis is the \gls{lulcu} \autocite{HeiGIT2024a}, a tool developed by the Climate Action focus group of the \gls{heigit} for \gls{lulcm} \autocite{HeiGIT2024}. The \gls{lulcu} utilizes the SegFormer architecture proposed by \textcite{Xie.Wang.ea2021}, a Transformer-based semantic segmentation model originally for RGB images, with adaptations to support multi-spectral satellite bands and a flexible framework to allow for easy integration of additional data sources.

% RQs
The research questions of this study outline the aim of integrating a road network into the feature space of a suitable \gls{dl} model and assessing the impact of integrated spatial contextual information on the model's performance with:\pagebreak

\begin{enumerate}[label={RQ \arabic*:}]
    \item \emph{Which attributes of the \gls{osm} road network can provide the model spatial contextual information regarding \gls{lulc} classes, and what is their relationship to adjacent \gls{lulc} classes derived from \gls{osm}?} 
    \item \emph{What is a suitable encoding method to integrate them into the feature space of the model?}
    \item \emph{How much does the integrated road network and its attributes impact the model's performance and resource consumption?}
\end{enumerate}

% Structure
The further structure of this thesis is organized specifically to address these research questions. First, \gls{lulcm} and the segmentation technique are explained, followed by an introduction to \glspl{dl} for a basic understanding of the training process of neural networks. Later, the background of the SegFormer is explained with a deep dive into attention mechanisms. The methodology outlines the research design used to answer the research questions, including a description of \gls{osm} and utilized datasets, the explorative relationship analysis between select road attributes and \gls{lulc} classes, and the model used as baseline. It further answers RQ 2 by showing four different approaches to integrate the road network into the feature space of the \gls{lulcu}. The results of the experiments are presented and discussed critically in regard to the research questions and similar studies afterwards. The thesis is concluded with an outlook on future research and a summary of the key findings.
