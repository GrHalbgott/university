\chapter{Discussion}
\label{sec:discussion}

The study was built around proofing the concept of integrating spatial contextual information into a semantic segmentation model for \gls{lulcm} in order to enhance its performance and resource consumption. Hypothetically, by adding information about the spatial context of \gls{lulc} classes using a road network, the model convergence should be supported and the overall resource consumption reduced alongside an increase of the performance, as the model can make more informed decisions about the \gls{lulc} classes. For this purpose, select attributes of the \gls{osm} road network, road types and speed limits, were encoded and integrated in the feature space of the \gls{lulcu} framework for \gls{lulcm}, which utilizes the SegFormer model for semantic segmentation. This resulted in four extensions, Extension 1 with a boolean road network without further encodings, Extension 2 with encoded road types, Extension 3 with encoded speed limit classes, and Extension 4 with a combination of Extensions 2 and 3. This section discusses the key findings and limitations of the study, implications for future research, as well as giving suggestions for improvement in order to enhance the generalizability and adaptability of the methodology.

%%%%%%%%%%
\section{Key Findings}

The results show that the \gls{osm} road network with its selected attributes road types and speed limits indeed provides spatial contextual information about adjacent \gls{lulc} classes. Integrating roads into the feature space of the model has an impact on the model's performance, resource consumption, and efficiency. In order to enlarge the impact of the road network on the model, the vanilla feature space of the \gls{lulcu} was altered, reducing the satellite imagery based input data from nine bands to five, creating a baseline to compare the results to. The values of the \gls{iou} for the baseline are similar to the performance achieved on the ADE20K and COCO-Stuff benchmark with the SegFormer, shown in the original paper of the SegFormer by \textcite{Xie.Wang.ea2021}. Although not directly comparable, this nevertheless means that even with a reduced feature space, the model still performs well, making the baseline a meaningful comparison basis.

All in all, by integrating road network data into the model, the performance is increased except for select cases, while \gls{gpu} and \gls{cpu} are utilized less, but over a longer runtime. When comparing the \gls{oa} and \gls{iou} of the extensions to the baseline on the same amount of data, the extensions outperform the baseline in all cases, showing that the model can make more informed decisions about the \gls{lulc} classes using the road network. Interestingly, the road geometries alone already increase the performance of the model. As the epochs needed for convergence are reduced in the extensions, they also help the model to converge earlier with less iterations through the data, therefore, the convergence also is supported. However, as the time needed per epoch and runtime are increased considerably for all extensions, even when utilizing \gls{gpu} and \gls{cpu} less, the resource consumption was not improved, meaning that the integration of the road network attributes primarily increases the model's performance while using more resources. The trade-off has to be rated by the possible users and their priority when enhancing models, as the performance and resource consumption do not scale proportionally. 

During the analysis of the results, it became clear that road types have generally a greater positive impact than speed limit classes, as the \gls{oa}, \gls{iou}, and \gls{ca} of the according Extension 2 is generally higher than Extension 3. The only exception is the \gls{lulc} classes forest, where the speed limits achieved a higher \gls{ca}. It should be noted that speed limit classes still improve the performance of the model, although usually to a lesser extent than road types. However, the combination of both road types and speed limits can have an even greater positive impact on the prediction of specific classes, as shown by Extension 4 for the \gls{lulc} classes built-up and permanent crops and the \gls{oa} and \gls{iou} scores for training and validation. But, only three of the six classes experienced an increase in performance in comparison to the other extensions, similar to Extension 2, emphasizing that most of the performance in Extension 4 derives from the road types. This aligns with the assumption stated above, that the speed limit classes will only provide a reduced amount of information, as the majority of the speed limits in the \gls{aoi} is unknown. Still, the impact of the existing speed limits is clearly noticeable, as the performance of the model is increased in the majority of the cases also in Extension 3. Therefore, increasing the complexity and providing as much information as possible does not necessarily lead to better performance, but has the potential to do so if the information has an according density and distribution. However, this comes with the cost of increased resource consumption, as the model has to process more information, which is especially noticeable in the runtime of the model.

According optimization techniques could be applied to reduce the resource consumption, making the model more efficient and sustainable. In section \ref{subsec:training}, the training of \glspl{dnn} was explained and the importance of optimization techniques emphasized. Optimization is one of the biggest challenges of \gls{dl}, as it is necessary in every aspect of the training process. This project was done based on a preset model framework, the \gls{lulcu}, without further adapting possible necessary optimizations for the adapted feature space, both for the baseline as well as for the extensions. Therefore, there is even more potential to be expected regarding the model's performance and efficiency with according tuning of the model's hyperparameters and optimization techniques. But, this has to be evaluated in separate studies, preferably by utilizing the vanilla \gls{lulcu} with its full feature space, as this would show the real potential of the road network as spatial contextual information and enhancement of the model.

%%%%%%%%%%
\section{Road Attributes \& \glsfmtshort{lulc} Classes}

The impact on performance and resource consumption varies between the extensions and the \gls{lulc} classes. The reason for this are the road network attributes utilized in this study themselves. The explorative relationship analysis showed that the tags of the selected attributes have a differentiated relationship towards adjacent \gls{lulc} classes on the one hand. But, the relationship is not as clear as expected on the other hand, as it shows overlaps between features of the attributes in their relationship distribution. Only the minority of the features has a distinct relationship towards only one \gls{lulc} class, most are distributed over multiple classes. Additionally, finding significant relationships or defining thresholds which divide the values into different strengths of relations is difficult without further statistical analyses. But still, the information found in the evaluated relationships still enhance the performance of the model, therefore, for this study, the chosen methodology was suitable.

Furthermore, the results show that the chosen encoding method, \gls{ohe}, was feasible as the information of the features was utilizable by the model. In \gls{ohe}, the features of the road network are rasterized and encoded into boolean tensor channels, which expand the model's feature space. Other encoding methods, including ordinal or binary encoding \autocite{Potdar.Pardawala.ea2017}, could also be utilized if the data meets their requirements, leading to different results for the model, as the values are offered differently to the model to learn from.

Nevertheless, encoding all features of the road network includes many negligible ones, which expand the feature space unnecessary as they do not provide usable information to the model. Also, many features have such a low share in the road network, that including them in the feature space, even when they have a comparably distinct distribution, does not provide the model with a learnable concept or relationship that it can apply for inference, as these roads are barely present in other \glspl{aoi}, for example London, as shown by \textcite{Alghanim.Jilani.ea2021}. This applies for both road types and speed limit classes. Therefore, to reduce the road network's complexity, feature engineering techniques were used. As the model evaluation shows, these were also feasible, as they effectively reduced the number of road types by half and categorized a variable number of different speed limits into 13 fixed categories.

As they are categorized into fixed ranges of speeds based on road classification systems, the categories of the speed limits could be used globally. But, the relationship analysis showed that the categories could be further reduced, as the distribution of the features is not distinct enough to justify the eight categories. This would reduce the complexity of the feature space, so that Extensions 3 and 4 could be more efficient while offering a similar level of information. Mentioned above, the class \enquote{unlimited} is only feasible in Germany and some other minor cases, and as the distribution is similar to \enquote{very\_high+}, these two could be merged. Also, the \gls{hca} showed that \enquote{walk} and \enquote{very\_low} are very similar as well, so merging these too would not reduce the information content of the feature space. Furthermore, the categories are derived from road classification systems of different countries, summarized by \textcite{Vitkiene.Puodziukas.ea2017}. Using a different system could lead to different categories, which could be more distinct or more general, depending on the system and the \gls{lulc} classes to predict. This could also be a way to reduce the complexity of the feature space, as the categories could be more distinct and therefore provide more information to the model. However, their feasibility would have to be evaluated in a separate study and the problem of a low mapping coverage of the speed limits still persists. In order to tackle this, merging multiple datasets with \gls{osm}, for example social media data or measured data from traffic monitoring stations as proposed by \textcite{Camargo.Bright.ea2020,Ludwig.Psotta.ea2023}, could be used to enhance the coverage of the speed limits. This would also enhance the generalizability of the model, as the speed limits would be more representative of the actual speed limits in the \gls{aoi}, maybe evn enhancing the relationship to adjacent \gls{lulc} classes due to their realistic behavior indirectly or directly based on the surrounding \gls{lulc} areas.

As for the road types, the complexity of the features can be reduced further by omitting more road types and by only keeping those with the strongest relation to \gls{lulc} classes, like \emph{residential} and \emph{track}, which both enhance the prediction of built-up areas due to their occurrence and avoidance of these areas, whereby \emph{track} additionally hints towards forests. \textcite{Atwal.Anderson.ea2022} additionally merged the types \emph{primary}, \emph{secondary}, and \emph{tertiary}, but these were not merged in this study due to their different importance in the road network. However, as their distribution over \gls{lulc} classes is similar, they additionally could be merged to reduce the complexity of the feature space. Additionally, increasing the threshold of the global share also reduces the number of road types and helps in generalizing the procedure only to the most important types. The merge of the road types \emph{motorway, trunk}, and their links could be seen as oversimplification, as especially the links have a different distribution than their counterparts. The background for this merge was the applied \gls{lcbb}, whose impact on the model cannot be directly evaluated. As maximal values were used for the buffers, it is possible that road greenery next to roads or grass patches inside and around intersections, like shown in figure \ref{fig:lcbb}, were included in the buffers. This could have a negative impact on the model, as the model could learn to predict roads as grass or vice versa, reducing the accuracy of the grass class.

However, the \gls{ca} of the grass class is already low in the baseline with an accuracy of only 41 \%, so the \gls{lcbb} cannot be the reason for the low performance of the model in the grass class. This probably is based on the reduced feature space and missing satellite imagery, which would be beneficial for the prediction of grass. When comparing the grass to the permanent crops class, which also has a low \gls{ca} of 43 \% in the baseline, there is a large difference in the extensions. The road network enhances the \gls{ca} of the permanent crops class by 8 to 24 \%p, while the performance for the grass class is reduced in all extensions up to 21 \%p, although arguably the performance of Extension 2 is only 2 \%p lower. Looking at the confusion matrices in appendix section \ref{app:conf_matrices}, it is visible that the grass class is confused primarily with the classes permanent crops and farmland, although in some cases forest and built-up are also predicted instead. Orchards, one of the tags of the class permanent crops, are often covered by grass, and farmlands also resemble grasslands in some of their growth states, so the confusion is partially understandable. Even adding vegetation indices as proposed by \textcite{Tzepkenlis.Marthoglou.ea2023} would enhance this prediction only slightly, as the cover is indeed grass. The model predicts the cover correctly, but wrong for the evaluation metric, as the \gls{lulc} class mapped in \gls{osm} is different. Given the additional low \gls{ca} of permanent crops, it is advisable to reconsider the \gls{lulc} classes in order to enhance the model's performance, as these two classes contribute to the lower \gls{oa} compared to the baseline. Suitable \gls{lulc} classes for natural areas could be, for example, forest, bare land, and vegetation, as proposed by \textcite{Basheer.Wang.ea2022}. This would enhance the model's performance, as the classes are more distinct and the model can learn more easily to differentiate between them. The impact of the road network would have to be evaluated separately, as the model has to learn new relationships between the road network and the \gls{lulc} classes.

In that regard, it is to be expected that for different \glspl{aoi}, different shares of the features are present, questioning the generalizability of the model. This is especially true for the road types, as the global distribution of road types is not the same as in the \gls{aoi} of the study. Furthermore, the chosen \gls{aoi} has a characteristic distribution of \gls{lulc} classes and therefore a special relationship between roads and \gls{lulc} classes. The characteristic border between the Odenwald and the Upper Rhine Rift Valley in the \gls{mahdrnk} region provokes a distinct road architecture, avoiding building paved roads in the forested areas with rougher relief and connecting the built-up areas in the valley with more major road types with higher speed limits. But still, as multiple regions with different characteristic distributions were merged into this \gls{aoi}, the overall distribution is representative at least for Germany. All these factors influence the relationship between road types and \gls{lulc} classes. Still, orienting on the global share of the road types is a good basis for reducing the complexity of the according road network to a feasible and utilizable amount.

%%%%%%%%%%
\section{Methodology}

As this study is a proof-of-concept of integrating spatial context into semantic segmentation models to improve the model's performance and efficiency, the concrete relationships between roads and \gls{lulc} classes are only secondary to the overall goal of the study. Nevertheless, the results show tendencies, which are rediscovered analogous in the model evaluation, and by using the road lengths rather than the number of roads, the chosen methodology was generally feasible and sufficient for the task at hand.

The relationships could be explored further in future studies, as the relationships are not as clear as expected and the study only takes into account the relative intersection of roads with adjacent \gls{lulc} classes. The methodology proposed by \textcite{Atwal.Anderson.ea2022} seems more suitable for a more elaborate analysis of the relationship between road networks and \gls{lulc} classes, as it includes a vectorized approach to assess the spatial relationship between the two. These vectors could also be encoded into the feature space of the model to give the model a more detailed understanding of the relationship between roads and \gls{lulc} classes. This would also enhance the generalizability of the model, as the vectors could be applied to different \glspl{aoi} and the model could learn the relationship between roads and \gls{lulc} classes in a more general way. However, to achieve this, the model has to be trained on a much larger dataset, preferably with globally distributed \glspl{aoi}, in order to quantify the relationship realistically and generalized. This would also help in tackling the loss behavior of the models, shown in figure \ref{fig:losses}, where another reason for the lower validation than training loss could be noisy input data and not enough data for a representative validation, which could both be fixed by a larger dataset.

The clustering method to find similarities between the patterns of the explorative relationship analysis in order to reduce the complexity of the road network is based on the \gls{hca}, paired with elaborate feature engineering. This requires domain knowledge and many explicit preprocessing steps. Although suitable for this study, others like k-means and DBSCAN \autocite{Montazeri.Lilienthal.ea2021}, maximum likelihood exploratory factor analysis \autocite{Petrakis.Norman.ea2021}, or the principal component analysis \autocite{Dharani.Sreenivasulu2019} are also proposed as suitable algorithms for effective automated dimensionality reduction. Utilizing them has the potential to offer other insights on the relationships between road attributes and \gls{lulc} classes, enhancing the explorative relationship analysis. However, as \gls{osm} comes with drawbacks like completeness and correctness, feature engineering is required nonetheless to ensure a suitable quality level and a successful utilization of the data. Also, researchers should keep in mind that some information about the \gls{lulc} of the area could be lost if road types are merged together with a not strong enough similarity in their distribution over \gls{lulc} classes. The according task determines the level of detail needed and the most suitable approach.

In that regard, the applied automated feature engineering steps for the road network, including disregarding and merging features as well as filling missing values, are based on domain knowledge, literature, and the \gls{osm} Wiki. However, as \gls{osm} has many anomalies due to wrong mapping, locally restricted information and regional differences \autocite{Ludwig.Fendrich.ea2021}, or even vandalism \autocite{Neis.Goetz.ea2012,Vargas-Munoz.Srivastava.ea2021}, the information of these is lost during this feature engineering procedure, as only few anomalies are included. Although many tools exist to assess the quality of the volunteered \gls{osm} \autocite{Minghini.Frassinelli2019,Moradi.Roche.ea2021,Schott.Zell.ea2024}, researches have to still \enquote{trust} the data of \gls{osm} to some extent. It is advised to base the feature engineering procedure on the \gls{osm} Wiki, where suggested tags and values are summarized, as including every anomaly would require a highly detailed data analysis and a similarly complex feature engineering process.

As for the data itself, the chosen attributes road types and speed limits were selected based on literature and similar studies. Other attributes could be utilized for such a project as well, primarily speed restrictions in specific timeframes or access restrictions. These could also hint towards adjacent \gls{lulc}, for example residential areas when a speed limit is applied only during night time hours or industrial areas where only specific vehicles are allowed. The problem here is the mapping completeness, which is even lower than for the tag \emph{maxspeed} and the speed tags, as well as the low number of occurrences altogether according to \gls{osm} Taginfo. Based on the list of global feature shares and their combinations in \textcite{OSMTaginfo2024}, another idea is to utilize road surface types (tag \emph{surface}), because they also could hint towards specific \gls{lulc} classes. For example, a cobblestone road could be an indicator for a historic city center, while a gravel road could be an indicator for a rural area. The mapping completeness here (25.61 \%) is much higher than for \emph{maxspeed} (7.40 \%) or \emph{access} (6.20 \%), so this could be a possible attribute to utilize. The problem here, on the other hand, is that the distribution of values is more biased, as 44.89 \% of the tag \emph{surface} has the value \emph{asphalt}. The question here, of course, is also if there are similar or even stronger distinct signals between the different features to adjacent \gls{lulc} areas, which is doubtful. A combination of multiple tags can, therefore, be a solution for the bias. However, that would have to be evaluated in a separate study especially in regard to the resource consumption of the model due to the possibly large feature space. Nevertheless, as seen in the evaluation of the extension, the best attribute of the road network to utilize is the road type, as it has the most positive influence on the model's performance and the highest mapping completeness globally, which is further increasing, as shown by \textcite{Anderson.Sarkar.ea2019,Barrington-Leigh.Millard-Ball2017,Herfort.Lautenbach.ea2021}.     

%%%%%%%%%%
\section{Objective \& Data Redundancy}

As mentioned above, this study functions as a proof-of-concept that integrating spatial contextual information enhances \gls{lulcm}. The example of using road networks as according information was chosen based on the idea that they are crucial parts of the landscape and constructed in mind of the adjacent \gls{lulc} areas. But, using roads to classify the adjacent \gls{lulc} areas would be redundant, as \gls{lulc} areas are used for constructing the roads and the roads are then used to classify the \gls{lulc} areas. In other words, it would be a model leakage where the same information is recycled and used again. However, this is based on the assumption that the mapping completeness of the \gls{osm} data is sufficient in the according \glspl{aoi}, regarding both the \gls{lulc} areas as well as the roads. Also, it is based on the assumption that both are correctly mapped, which sometimes is not the case, as \gls{osm} includes many anomalies and is often based on volunteered action, possible with missing domain knowledge. \textcite{Barrington-Leigh.Millard-Ball2017} state that the road network, which is the main emphasis of \gls{osm} (hence the name), usually has a higher mapping completeness than other features, including \gls{lulc}. Also, there is a large bias in mapping completeness of additional features, which concentrate on very densely or sparsely populated areas, as shown by \textcite{Herfort.Lautenbach.ea2021}. Furthermore, utilizing \gls{osm} data always comes with additional drawbacks, including currentness and correctness, which applies also to \gls{lulc} areas.

Yet, as the main task of \gls{ml} models is to utilize learned concepts and models for inference on unknown data \autocite{Sarker2021,Shinde.Shah2018}, completeness is the drawback that has to be tackled the most. In order to tackle the completeness of roads and their attributes in specific \glspl{aoi}, merging data from different sources could be a solution, for example authoritative datasets or the Microsoft Roads dataset, as proposed by \textcite{Anderson.Sarkar.ea2019,Srivastava.VargasMunoz.ea2019,Vargas-Munoz.Srivastava.ea2021}. Moreover, multiple studies introduced \gls{ml} based road type classification algorithms \autocite{Alghanim.Jilani.ea2021,Zhao.Ning.ea2023} using remote sensing and street-view imagery, which, in combination with road detection algorithms \autocite{Abdollahi.Pradhan.ea2020,Chen.Deng.ea2022}, could provide a fully automated \enquote{road detection to \gls{lulcm}} approach. Additionally, average speeds could also be estimated using \gls{dl} approaches, as proposed by \textcite{Keller.Gabriel.ea2020}, further enriching or filling missing data. These methods could tackle the problem of applying this methodology to areas with a low mapping completeness, as roads could be extracted and classified by utilizing \gls{dl} approaches. Both could then be used to enhance the prediction of surrounding \gls{lulc} areas, which in turn could function as a verification for the classified road types. In such cases, utilizing road networks for \gls{lulcm} can provide a real case scenario where \gls{lulcm} of unknown areas can be enhanced with \gls{dl} generated roads. However, such an elaborated approach has to be evaluated in future studies, as the complexity of the model and the resource consumption could increase considerably, providing a bad trade-off between efficiency and performance. In that regard, other model architectures than the SegFormer could also be used for this task, as the SegFormer is not the only model suitable for semantic segmentation. \textcite{Tzepkenlis.Marthoglou.ea2023}, for example, propose a U-Net with a temporal attention encoder, which outperforms the SegFormer in terms of performance and efficiency. In order to measure the efficiency of the according model effectively, \textcite{Desislavov.Martinez-Plumed.ea2023,Getzner.Charpentier.ea2023} propose using Floating Point Operations as a measure of computational complexity, independent from machine configuration. \textcite{Mehlin.Schacht.ea2023} further propose multiple other metrics to measure carbon emission, energy usage, and computational complexity, in order to enhance the evaluation of the model's efficiency and sustainability, which should further be considered in future studies.

All in all, the methodology was suitable for the study and the results show that the integration of spatial contextual information into a semantic segmentation model for \gls{lulcm} is feasible. It enhances the model's performance and offers possible improvements to resource consumption and efficiency, if according optimization techniques are implemented. Therefore, the feasibility of the concept is proven.
